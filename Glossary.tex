\hypertarget{preface}{%
\chapter*{Glossaire anglais-français}\label{glossary}}

\addcontentsline{toc}{chapter}{Glossaire anglais-français}

\begin{longtable}{p{0.45\textwidth} p{0.35\textwidth} p{0.2\textwidth}} 
\textbf{Terme français} & \textbf{English term}  & \textbf{Acronyme} \\ \endfirsthead
\textbf{Terme français} & \textbf{English term}  & \textbf{Acronyme} \\ \endhead
abandon & dropout &  \\ 
accroche & anchor &  \\ 
action & action &  \\ 
aire sous la courbe ROC & area under the ROC curve & AUC \\ 
algorithme de tri & sorting algorithm &  \\ 
algorithme espérance-maximisation & expectation-maximization algorithm &  \\ 
algorithme génétique & genetic algorithm &  \\ 
algorithmique & analysis of algorithms &  \\ 
allocation de Dirichlet latente & latent Dirichlet allocation &  \\ 
analyse en composantes principales & principal component analysis & ACP \\ 
apprendre à classer & learning to rank &  \\ 
apprentissage à base de modèles & model-based learning &  \\ 
apprentissage actif & active learning &  \\ 
apprentissage auto-supervisé & self-supervised learning &  \\ 
apprentissage de concepts & zero-shot learning & ZSL \\ 
apprentissage de métriques & metric learning &  \\ 
apprentissage incrémental & incremental learning &  \\ 
apprentissage machine & machine learning & ML \\ 
apprentissage non supervisé & unsupervised learning &  \\ 
apprentissage PAC & Probably approximately correct learning & PAC learning \\ 
apprentissage par instanciation & instance-based learning &  \\ 
apprentissage par liste & listwise learning &  \\ 
apprentissage par paire & pairwise learning &  \\ 
apprentissage par renforcement & reinforcement learning & RL \\ 
apprentissage peu profond & shallow learning &  \\ 
apprentissage ponctuel & pointwise learning &  \\ 
apprentissage profond & deep learning & DL \\ 
apprentissage semi-supervisé & semi-supervised learning & SSL \\ 
apprentissage séquence vers séquence & sequence-to-sequence learning & seq2seq learning \\ 
apprentissage supervisé & supervised learning &  \\ 
apprentissage unique & one-shot learning &  \\ 
approximation et projection uniforme de variétés & uniform manifold approximation and projection & UMAP \\ 
arbre de décision & decision tree &  \\ 
architecture avec mécanismes d’attention & architecture with attention &  \\ 
astuce du noyau & kernel trick &  \\ 
augmentation de données & data augmentation &  \\ 
augmentation de données & data augmentation &  \\ 
auto-apprentissage & self-learning &  \\ 
auto-encodeur & autoencoder &  \\ 
auto-encodeur débruitant & denoising autoencoder &  \\ 
bagging & bagging &  \\ 
biais & bias &  \\ 
bloc & fold &  \\ 
boosting & boosting &  \\ 
bornage de la norme du gradient & gradient clipping &  \\ 
bruit & noise &  \\ 
bruit blanc gaussien & Gaussian noise &  \\ 
bulle de filtre & filter bubble &  \\ 
calcul de la moyenne & meta-model &  \\ 
caractéristique & feature &  \\ 
caractéristique catégorielle & categorical feature &  \\ 
cellule mémoire & memory cell &  \\ 
centroïde & centroid &  \\ 
champs aléatoires conditionnels & conditional random fields & CRF \\ 
cible & target &  \\ 
classe & classe &  \\ 
classeur & ranker &  \\ 
classification & classification &  \\ 
classification binaire & binary classification &  \\ 
classification hiérarchique & hierarchical clustering &  \\ 
classification monoclasse & one-class classification, unary classification, class modeling &  \\ 
classification multiclasses & multiclass classification, multinomial classification &  \\ 
classification multi-labels & multi-label classification &  \\ 
codomaine & codomain &  \\ 
compromis biais-variance & bias-variance tradeoff &  \\ 
connexion saute-couche & skip connection &  \\ 
convexe & convex &  \\ 
convolution & convolution &  \\ 
couche & layer &  \\ 
couche cachée & hidden layer &  \\ 
couche en entonnoir & bottleneck layer &  \\ 
couche en goulot d’étranglement & bottleneck layer &  \\ 
décalage & stride &  \\ 
décodeur & decoder &  \\ 
dérivation & differentiation &  \\ 
dérivée & derivative &  \\ 
dérivée partielle & partial derivative &  \\ 
désactivation & dropout &  \\ 
descente de gradient & gradient descent &  \\ 
descente de gradient stochastique & stochastic gradient descent &  \\ 
déséquilibré & imbalanced &  \\ 
détection d’anomalies & anomaly detection &  \\ 
détection de nouveautés & novelty detection &  \\ 
détection de valeurs aberrantes & outlier detection &  \\ 
déterminant & determinant &  \\ 
différentiation & differentiation &  \\ 
dimension & dimension &  \\ 
disparition du gradient & vanishing gradient &  \\ 
distance euclidienne & Euclidean distance &  \\ 
distribution de probabilités & probability distribution &  \\ 
distribution normale multivariée & multivariate normal distribution &  \\ 
distribution normale standard & standard normal distribution &  \\ 
domaine & domain &  \\ 
donnée multimodale & multimodal data &  \\ 
données d’entraînement & training set &  \\ 
données de test & test set &  \\ 
données de validation & validation set &  \\ 
écart-type & standard deviation &  \\ 
échantillon aléatoire & random sample &  \\ 
échantillonnage négatif & negative sampling &  \\ 
échantillonnage synthétique adaptatif & adaptive synthetic sampling method & ADASYN  \\ 
élagage & pruning &  \\ 
empilage & stacking &  \\ 
encodage à chaud & one-hot encoding &  \\ 
encodeur & encoder &  \\ 
ensemble & set &  \\ 
ensemble flou & fuzzy set &  \\ 
entièrement connecté & fully-connected &  \\ 
entraînement & training &  \\ 
entropie & entropy &  \\ 
entropie croisée appliquée aux ensembles flous & fuzzy set cross-entropy &  \\ 
entropie croisée binaire & binary cross-entropy &  \\ 
environnement & environment &  \\ 
erreur d'apprentissage & empirical risk &  \\ 
erreur quadratique & squared error &  \\ 
erreur quadratique moyenne intégrée & mean integrated squared error & MISE \\ 
espace de redescription & transformed space &  \\ 
espérance & outlier detection &  \\ 
estimateur non biaisé & unbiased estimator &  \\ 
estimation de densité & density estimation &  \\ 
estimation par omission exemple par exemple & leave one out estimate &  \\ 
état & state &  \\ 
étiquette & tag &  \\ 
exemple & example &  \\ 
explosion du gradient & exploding gradient &  \\ 
extraction d’entités nommées & named entity extraction &  \\ 
extraction de caractéristiques & feature engineering &  \\ 
faux négatif & false negative & FN \\ 
faux positif & false positive & FP \\ 
fenêtre glissante & moving window &  \\ 
feuille & leaf node &  \\ 
filtrage basé sur le contenu & content-based filtering &  \\ 
filtrage collaboratif & collaborative filtering &  \\ 
filtre & filter &  \\ 
fonction à valeurs vectorielles & vector function &  \\ 
fonction d’activation & activation function &  \\ 
fonction d’efficacité du récepteur & receiver operating characteristic &  \\ 
fonction de base radiale & radial basis function & RBF \\ 
fonction de coût & cost function &  \\ 
fonction de coût triplet & loss triple, triplet loss &  \\ 
fonction de densité de probabilité & probability density function & PDF \\ 
fonction de masse & probability mass function & PMF \\ 
fonction de perte & loss function &  \\ 
fonction de similarité & similarity function &  \\ 
fonction d'unité de rectification linéaire & rectified linear unit function &  \\ 
fonction imbriquée & nested function &  \\ 
fonction logistique & standard logistic function &  \\ 
fonction objectif & objective function &  \\ 
fonction porte & gate function &  \\ 
fonction scalaire & scalar function &  \\ 
fonction sigmoïde & sigmoid function &  \\ 
fonction softmax & softmax function &  \\ 
fonction tangente hyperbolique & hyperbolic tangent function &  \\ 
force d’appartenance & membership strength &  \\ 
force de prédiction & prediction strength &  \\ 
forêts aléatoires & random forest &  \\ 
forêts aléatoires & random forest &  \\ 
frontière de décision & decision boundary &  \\ 
généralisation & generalization &  \\ 
gradient & gradient &  \\ 
gradient boosting & gradient boosting &  \\ 
graphe & graph &  \\ 
graphe de dépendances & dependency graph &  \\ 
grille de recherche & grid search &  \\ 
groupe & cluster &  \\ 
groupement des données par classes & binning, bucketing &  \\ 
groupement spatial d’applications basé sur la densité avec bruit & density-based spatial clustering of applications with noise & DBSCAN \\ 
groupes mutuellement exclusifs & hard clustering &  \\ 
groupes mutuellement non-exclusifs & soft clustering &  \\ 
hyperparamètre & hyperparameter &  \\ 
image & codomain &  \\ 
imputation de données & data imputation &  \\ 
inégalité triangulaire & triangle inequality &  \\ 
intégrale & integral &  \\ 
interruption prématurée & early stopping &  \\ 
intersection & intersection &  \\ 
intervalle & interval &  \\ 
intervalle de confiance & confidence interval &  \\ 
intervalle ouvert & open interval &  \\ 
inverse & inverse &  \\ 
itération & epoch &  \\ 
jeu à somme nulle & zero-sum game &  \\ 
justesse & accuracy &  \\ 
justesse sensible au coût & cost-sensitive accuracy &  \\ 
k plus proches voisins & k-nearest neighbors & KNN \\ 
k-moyennes & k-means &  \\ 
label & label &  \\ 
labellisation de séquences & sequence labeling &  \\ 
labellisé & labeled &  \\ 
log-vraisemblance & log-likelihood &  \\ 
loi de Gauss & Gaussian distribution &  \\ 
loi de probabilité a priori & prior &  \\ 
loi de probabilité marginale & prior &  \\ 
loi normale & normal distribution &  \\ 
machine à vecteurs de support & support vector machine & SVM \\ 
machine de factorisation & factorization machine &  \\ 
marge & padding &  \\ 
marge dure & hard margin &  \\ 
marge souple & soft margin &  \\ 
matrice & matrix &  \\ 
matrice creuse & sparse matrix &  \\ 
matrice de coappartenance & co-membership matrix &  \\ 
matrice de confusion & confusion matrix &  \\ 
matrice diagonale & diagonal matrix &  \\ 
matrice identité & identity matrix &  \\ 
maximum a posteriori & maximum a posteriori & MAP \\ 
maximum de vraisemblance & maximum likelihood &  \\ 
mégadonnées & big data &  \\ 
méta-modèle & meta-model &  \\ 
méthode de la moyenne des silhouettes & average silhouette method &  \\ 
méthode de Monte-Carlo par chaîne de Markov & Markov Chain Monte Carlo method &  \\ 
méthode du coude & elbow method &  \\ 
méthodes ensemblistes & ensemble models, ensemble learning &  \\ 
mini-lots & minibatch &  \\ 
minimum global & global minimum &  \\ 
minimum local & local minimum &  \\ 
modèle creux & sparse model &  \\ 
modèle de base & base model &  \\ 
modèle de mélange gaussien & gaussian mixture model &  \\ 
modèle ensembliste & ensemble model &  \\ 
modèle faible & weak model &  \\ 
modèle gaussien & one-class Gaussian &  \\ 
modèle graphique probabiliste & probabilistic graphical model &  \\ 
modèle linéaire & linear model &  \\ 
modèle linéaire généralisé & generalized linear model & GLM \\ 
modèle non-paramétrique & nonparametric model &  \\ 
modèle paramétrique & parametric model &  \\ 
modèle statistique & statistical model &  \\ 
modélisation de sujet & topic modeling &  \\ 
moment & momentum &  \\ 
moyenne & mean, average, expected value &  \\ 
moyenne de l’échantillon & sample mean &  \\ 
moyenne de la précision moyenne & mean average precision & MAP \\ 
moyenne des erreurs au carré & mean squared error & MSE \\ 
multiplicateurs de Lagrange & Lagrange multipliers &  \\ 
nœud & node &  \\ 
non-labellisé & unlabeled &  \\ 
non-négativité & nonnegativity &  \\ 
normalisation & normalization &  \\ 
normalisation par lot & batch normalization &  \\ 
noyau & kernel &  \\ 
noyau gaussien & Gaussian kernel &  \\ 
objectif & objective &  \\ 
optimisation bayésienne des hyperparamètres & Bayesian hyperparameter optimization &  \\ 
paramètre & parameters &  \\ 
partitionnement spectral & spectral clustering &  \\ 
patch & patch &  \\ 
perceptron multicouches & multilayer perceptron & MLP \\ 
perte hinge & hinge loss &  \\ 
plongement & embedding &  \\ 
plongement de mots & word embedding &  \\ 
plongement lexical & word embedding &  \\ 
poids & weight &  \\ 
point de contrôle & checkpoint &  \\ 
politique & policy &  \\ 
porte d'oubli & forget gate &  \\ 
pouvoir de prédiction & predictive power &  \\ 
précision & precision &  \\ 
prédicteur & predictor &  \\ 
prédicteur faible & weak learner &  \\ 
processus gaussien & gaussian process &  \\ 
produit scalaire & dot product &  \\ 
programmation quadratique & quadratic programming &  \\ 
rang & rank &  \\ 
rappel & recall &  \\ 
recherche aléatoire & random search &  \\ 
recherche arrière & backtracking &  \\ 
récompense & reward &  \\ 
réduction de dimension & dimensionality reduction &  \\ 
réglage des hyperparamètres & hyperparameters tuning &  \\ 
règle de Bayes & Bayes’ Rule &  \\ 
régresseur fort & strong regressor &  \\ 
régression & regression &  \\ 
régression linéaire & linear regression &  \\ 
régression logistique & logistic regression &  \\ 
régression par noyau & kernel regression &  \\ 
régression polynomiale & polynomial regression &  \\ 
regroupement & clustering &  \\ 
regroupement (couche d'un réseau de neurones) & pooling &  \\ 
régularisation & regularization &  \\ 
requête par votes & query by committee &  \\ 
réseau adverse génératif & generative adversarial network &  \\ 
réseau bayésien & belief network &  \\ 
réseau de croyances & belief network &  \\ 
réseau de neurones & neural network & NN \\ 
réseau de neurones à échelle & ladder network &  \\ 
réseau de neurones à propagation avant & feed-forward neural network & FFNN \\ 
réseau de neurones convolutif & convolutional neural network & CNN \\ 
réseau de neurones profond & deep neural network &  \\ 
réseau de neurones récurrent & recurrent neural network & RNN \\ 
réseau de neurones récurrent à portes & gated recurrent neural network &  \\ 
réseau de neurones récurrent avec mécanisme d’attention & recurrent neural network with attention &  \\ 
réseau de neurones récurrent bidirectionnel & bi-directional recurrent neural network &  \\ 
réseau de neurones récurrent séquence à séquence & sequence-to-sequence recurrent neural network &  \\ 
réseau de neurones récursif & recursive neural network &  \\ 
réseau de neurones résiduel & residual neural network &  \\ 
réseau de neurones siamois & siamese neural network & SNN \\ 
réseau récurrent à mémoire court et long terme & long short-term memory network & LSTM \\ 
résidu & residual &  \\ 
rétropropagation & backpropagation &  \\ 
rétropropagation à travers le temps & backpropagation through time &  \\ 
risque empirique & empirical risk &  \\ 
sac de mots & bag of words &  \\ 
scalaire & scalar &  \\ 
score d’appartenance & membership score &  \\ 
sélection de variables & feature selection &  \\ 
semi-définie positive & positive semidefinite &  \\ 
séparateur à vaste marge & support vector machine & SVM \\ 
seuil de décision & decision threshold &  \\ 
similarité cosinus & cosine similarity &  \\ 
skip-gram & skip-gram &  \\ 
softmax hiérarchique & hierarchical softmax &  \\ 
solveur & solver &  \\ 
sous-apprentissage & underfitting &  \\ 
sous-échantillonnage & undersampling &  \\ 
standardisation & standardization &  \\ 
statistique & statistic &  \\ 
statistique de l’échantillon & sample statistic &  \\ 
statistique d'écart & gap statistic &  \\ 
surapprentissage & overfitting &  \\ 
suréchantillonnage & oversampling &  \\ 
suréchantillonnage minoritaire synthétique & synthetic minority oversampling technique & SMOTE  \\ 
symétrie & symmetry &  \\ 
système de recommandation & recommender system &  \\ 
taux d’apprentissage & learning rate &  \\ 
taux de faux positifs & false positive rate &  \\ 
taux de vrais positifs & true positive rate &  \\ 
théorème de Bayes & Bayes’ Theorem &  \\ 
théorème de dérivation des fonctions composées & chain rule &  \\ 
théorie des matrices & matrix theory &  \\ 
tirage avec remise & sampling with replacement &  \\ 
token & token &  \\ 
transfert d’apprentissage & transfer learning &  \\ 
transposé & transpose &  \\ 
tri à bulles & bubble sort &  \\ 
tri par propagation & bubble sort &  \\ 
un contre tous & one versus rest &  \\ 
union & union &  \\ 
unité & unit &  \\ 
unité à portes & gated unit & GU \\ 
unité récurrente à portes & gated recurrent unit & GRU \\ 
valeur aberrante & outlier &  \\ 
validation croisée & cross-validation &  \\ 
validation croisée à 5 blocs & five-fold cross-validation &  \\ 
variable aléatoire & random variable &  \\ 
variable aléatoire continue & continuous random variable &  \\ 
variable aléatoire discrète & discrete random variable &  \\ 
variance & variance &  \\ 
vecteur & vector &  \\ 
vecteur de caractéristiques & feature vector &  \\ 
vecteur de support & support vector &  \\ 
vision artificielle & computer vision &  \\ 
volume & volume &  \\ 
vote majoritaire & majority vote &  \\ 
vrai négatif & true negative & TN \\ 
vrai positif & true positive & TP \\ 
vraisemblance & likelihood &  \\
\end{longtable}
